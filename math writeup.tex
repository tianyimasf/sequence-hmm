\documentclass[12pt]{article}
\usepackage[left=1in, right=1in, top=1in, bottom=1in]{geometry}
\usepackage{amsmath}
\usepackage{amsfonts}
\usepackage{mathtools}
\newcommand{\R}{\mathbb{R}}
\setlength{\parindent}{0pt}

\begin{document}
\title{CSC246 Sequence Project Math writeup}
\author{Tianyi Ma}
\maketitle

Calculating $\alpha_t(j)$:\\
1. Initialization:
\begin{align*}
\alpha_1(j) = \pi_jb_j(o_1) 1 \leq j \leq N
\end{align*}
2. Recursion:
\begin{align*}
\alpha_t(j) = \sum_{i=1}^N \alpha_{t-1}(i)a_{ij}b_j(o_t); 1 \leq j \leq N, 1 < t \leq T
\end{align*}
3. Termination:
\begin{align*}
P(O|\lambda) = \sum_{i=1}^N \alpha_T(i)
\end{align*}

$\pi_j$ - the initial probability of hidden state $j$\\
$a_{ij}$ - the transition probability from state $i$ to $j$\\
$b_j(o_t)$ - the emission probability of state $j$ to the $t$-th observation\\
$N$ - the number of possible hidden states\\
$T$ - the total number of tokens in the sequence\\
$P(O|\lambda)$ - the probability of the sample sequence based on current parameters\\

Calculating $\beta_t(i)$:\\
1. Initialization:
\begin{align*}
\beta_T(i) = 1; \text{    } 1 \leq i \leq N
\end{align*}
2. Recursion:
\begin{align*}
\beta_t(i) = \sum_{j=1}^N a_{ij}b_j(o_{t+1})\beta_{t+1}(j); \text{    } 1 \leq i \leq N, 1 \leq t < T
\end{align*}
3. Termination:
\begin{align*}
P(O|\lambda) = \sum_{j=1}^N \pi_jb_j(o_1)\beta_1(j)
\end{align*}

EM-algorithm:\\
E-step:
\begin{align*}
\gamma_t(j) &= \frac{\alpha_t(j)\beta_t(j)}{\alpha_T(q_F)}  \text{    } \forall t \text{ and } j\\
\xi_t(i,j) &= \frac{\alpha_t(j)a_{ij}b_j\beta_{t+1}(j)}{\alpha_T(q_F)}  \text{    } \forall t, i,\text{and } j
\end{align*}
M-step:
\begin{align*}
\hat{a}_{ij} &= \frac{\sum_{t=1}^{T-1}\xi_t(i,j)}{\sum_{t=1}^{T-1}\sum_{k=1}^N\xi_t(i,k)}\\
\hat{b}_j(v_k) &= \frac{\sum_{t=1 s.t. O_t = v_k}^T \gamma_t(i,j)}{\sum_{t=1}^T\gamma_t(j)}
\end{align*}

Two questions:\\
1. How to compute the log likelihood?\\
2. What is $q_F$ in the E-step?


\end{document}